%%% Exercise 1.7.4
\begin{exrc}[4]
    Let \(e\) be a character on \(G = \Z(N)\).
    Show the there exists a unique integer \(0 \leq l \leq N - 1\) so that
    \begin{equation*}
        e(k) = e_l(k) = e^{2 \pi i l k / N} \quad \text{for all \(k \in \Z(N)\)}.
    \end{equation*}
    Conversely, every function of this type is a character on \(\Z(N)\).
    Deduce that \(e_l \mapsto l\) defines an isomorphism from \(\hat{G}\) to \(G\).

\begin{soln}
    Since \((N + 1) \cdot 1 = 1\) and by definition of a character, we see
    \begin{equation*}
        e(1) = e((N + 1) \cdot 1) = e(1)^{N + 1},
    \end{equation*}
    where \((N + 1) \cdot 1\) is \(1\) added \(N + 1\) times.
    Therefore, \(e(1)^N = 1\), i.e., \(e(1)\) is an \(N\)'th root of unity,
    so \(e(1) = e^{2 \pi i l / N}\) for some \(0 \leq l < N\).
    Since \(k = k \cdot 1\), we see
    \begin{equation*}
        e(k) = e(1)^k = e^{2\pi i l k / N},
    \end{equation*}
    as desired.
    Also every function \(e_l(k) = e^{-2 \pi i l k / N}\)
    is a character on \(\Z(N)\);
    it is well defined since \(e_l(k) = e_l(k + N)\),
    and it respects the group operation since \(e_l(k_1 + k_2) = e_l(k_1)e^l(k_2)\).

    The map \(e_l \mapsto l\) is a homomorphism from \(\hat{G}\) to \(G\)
    since \(e_{l_1} e_{l_2} = e_{l_1 + l_2}\).
    This map is clearly bijective, therefore \(\hat{G}\) and \(G\) are isomorphic.

\end{soln}
\end{exrc}
