%%% Exercise 1.2.18
\begin{exrc}[18]
    If \(P_r(\theta)\) denotes the Poisson kernel, show that the function
    \begin{equation*}
        u(r, \theta) = \frac{\partial P_r}{\partial \theta},
    \end{equation*}
    defined for \(0 \leq r < 1\) and \(\theta \in \R\), satisfies:
    \begin{enumerate}[(i)]
    \item
        \(\bigtriangleup u = 0\) in the disc.
    \item
        \(\lim_{r \to 1} u(r, \theta) = 0\) for each \(\theta\).
    \end{enumerate}
    However, \(u\) is not identically zero.

\begin{soln}
    Recall that \(P_r = \sum_{n = -\infty}^{\infty} r^{|n|} e^{in\theta}\). 
    Differentiating each term with respect to \(\theta\) gives
    \begin{equation*}
        \sum_{n = -\infty}^{\infty} in r^{|n|} e^{in\theta}
        = \sum_{n = 1}^{\infty} inr^n (e^{in\theta} - e^{-in\theta})
        = \sum_{n = 1}^{\infty} -2nr^n \sin n\theta.
    \end{equation*}
    Since this sum converges uniformly in \(\theta\) on any closed disc
    contained in the unit disc,
    the above sum is in fact the partial derivative of the Poisson kernel, i.e.,
    \begin{equation*}
        u(r, \theta) = \sum_{n = 1}^{\infty} -2nr^n \sin n\theta.
    \end{equation*}
    Like the proof of part (i) of Theorem 5.7, by considering any \(\rho < 1\),
    the series for \(u\) can be differentiated twice term by term (in either variable)
    and the differentiated series is uniformly convergent in the disc of radius \(\rho\).
    Therefore, \(u\) can be twice differentiated with respect to both \(r\) and \(\theta\)
    by differentiating the series term by term.
    Treating the case \(n = 1\) and \(n > 1\) separately and using polar coordinates, 
    it is easy to show the Laplacian \(\bigtriangleup\) of each term in the series is 0,
    hence \(\bigtriangleup u = 0\).  This shows point (i).

    Point (ii) in the special case that \(\theta = 0\) is obvious by considering the 
    series expression of \(u\).
    Otherwise, suppose \(\theta \neq 0\).
    By considering the closed form of \(P_r\), we obtain
    \begin{equation*}
        u(r, \theta) = \frac{\partial}{\partial \theta} \frac{1 - r^2}{1 - 2r \cos \theta + r^2}
        = \frac{2r (r^2 - 1) \sin x}{(r^2 - 2r \cos x + 1)^2}.
    \end{equation*}
    Taking \(r \to 1\) gives a zero numerator and nonzero denominator,
    hence 
    \begin{equation*}
        \lim_{r \to 1} u(r, \theta) = 0,
    \end{equation*}
    which completes the proof of point (ii).

    However, \(u\) is not identically zero.
    For example, by using the closed form of \(u\), we see
    \begin{equation*}
        u(1 / 2, \theta) = -\frac{3 \sin \theta}{(5 / 4 - 2 \cos \theta)^2}.
    \end{equation*}

\end{soln}
\end{exrc}
