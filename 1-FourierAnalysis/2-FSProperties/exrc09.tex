%%% Exercise 1.2.9
\begin{exrc}[9]
    Let \(f(x) = \chi_{[a, b]}(x)\) be the characteristic function of the interval
    \([a, b] \subset [-\pi, \pi]\).
    \begin{enumerate}[(a)]
    \item
        Show that the Fourier series of \(f\) is given by
        \begin{equation*}
            f(x) \sim \frac{b - a}{2\pi} + \sum_{n \neq 0}^{} \frac{e^{-ina} - e^{-inb}}{2 \pi i n}e^{e^{inx}}.
        \end{equation*}

    \item
        Show that if \(a \neq -\pi\) and \(b \neq \pi\), and \(a \neq b\),
        then the Fourier series does not converge absolutely for any \(x\).

    \item
        However, prove that the Fourier series converges at every point \(x\).
        What happens if \(a = -\pi\) and \(b = \pi\)?
    \end{enumerate}

\begin{soln}
    \begin{enumerate}[(a)]
    \item
        Clearly
        \begin{equation*}
            \hat{f}(0) = \frac{1}{2\pi} \int_{a}^{b}  \ud x = \frac{b - a}{2\pi}
        \end{equation*}
        and
        \begin{equation*}
            \hat{f}(n) = \frac{1}{2\pi} \int_{a}^{b} e^{-inx} \ud x
            = \frac{e^{-ina} - e^{-inb}}{2 \pi i n}.
        \end{equation*}
        for \(n \neq 0\).

    \item
        Let \(\vartheta = \frac{b + a}{2}\),
        Since \(\abs{e^{in \vartheta}} = 1\) and
        \begin{equation*}
            \frac{e^{-ina} - e^{-inb}}{2i} e^{in \vartheta} = \sin n \theta_0,
        \end{equation*}
        it suffices to show that
        \begin{equation*}
            \sum_{n \neq 0}^{} \frac{\sin n\theta_0}{n}
        \end{equation*}
        does not converge absolutely.
        For that, since \(\sum_{}^{} 1 / n\) does not converge absolutely,
        it suffices to show that \(\limsup_{n \to \infty} |\sin n\theta_0| > 0\).

        First suppose that \(b - a\) is rational, i.e.,
        \(\theta_0 = \frac{p}{q} \pi\).
        By assumption on \(a\) and \(b\), we see that \(0 < p / q < 1\), i.e., \(\sin \theta_0 \neq 0\).
        Therefore, if \(n = 2kq + 1\) for any integer \(k\),
        since \(\sin x\) is \(2\pi\)-periodic,
        we get
        \begin{equation*}
            \sin n \theta_0 = \sin \theta_0 \neq 0.
        \end{equation*}
        Thus \(\limsup_{n \to \infty} |\sin n\theta_0| \geq |\sin \theta_0| > 0\)
        when \(\theta_0\) is a rational multiple of \(\pi\).

        If on the other hand \(b - a\) is irrational,
        then we can use the following fact;
        if \(\alpha\) is irrational,
        then \(\angl{n \alpha}\) is equidistributed, hence dense, in \([0, 1]\),
        where \(\angl{x}\) denotes the fractional part of \(x\).
        (This is proven in Chapter 4.)
        The above result when scaled by \(2\pi\)
        implies that \(2 \pi n \alpha \pmod {2\pi}\) is dense on the circle.
        Therefore, you can always find an \(n\) as large as you want such that
        \(n \theta_0 \pmod{2\pi}\) is sufficiently close to \(\pi / 2\).
        This proves that \(\limsup_{n \to \infty} |\sin n\theta_0| = 1 > 0\).

    \item
        However, the series converges for all \(x\) by Dirichlet's test for convergence
        (Exercise 7) with \(a_n = 1 / n\) and \(b_n = (e^{-ina} - e^{-inb}) e^{inx}\)
        so \(|b_n| \leq 2\).
        If \(a = -\pi\) and \(b = \pi\),
        then each term in the summation is zero, hence the Fourier series \(f(x) \sim 1\).
    \end{enumerate}

\end{soln}
\end{exrc}
