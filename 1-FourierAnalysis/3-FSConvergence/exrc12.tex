%%% Exercise 1.3.12
\begin{exrc}[12]
    Prove that
    \begin{equation*}
        \int_{0}^{\infty} \frac{\sin x}{x} \ud x = \frac{x}{2}.
    \end{equation*}

\begin{soln}
    Since \(\int_{-\pi}^{\pi} e^{inx} \ud x = 0\) for any \(n \neq 0\),
    the integral of the Dirichlet kernel is
    \begin{equation*}
        \int_{-\pi}^{\pi} D_N(x) \ud x
        = \int_{-\pi}^{\pi} \sum_{n = -N}^{N} e^{-inx} \ud x
        = \sum_{n = -N}^{N} \int_{-\pi}^{\pi} e^{-inx} \ud x
        = \int_{-\pi}^{\pi} 1 \ud x
        = 2\pi.
    \end{equation*}
    Therefore,
    \begin{align*}
        2\pi
        &= \int_{-\pi}^{\pi} D_n(x) \ud x \\
        &= \int_{-\pi}^{\pi} \frac{\sin((N + 1 / 2)x)}{\sin(x / 2)} \ud x \\
        &= \int_{-\pi}^{\pi} \sin\br{(N + 1/2)x} \sqbr{\frac{1}{\sin(x/2)}
        - \frac{2}{x} + \frac{2}{x}} \ud x.
    \end{align*}
    Define
    \begin{equation*}
        g(x) = \frac{1}{\sin(x / 2)} - \frac{2}{x}.
    \end{equation*}
    Clearly, \(g\) is continuous on \([-\pi, \pi] \setminus \set{0}\),
    and we can extend \(g\) to be continuous at 0 by defining \(g(0) = 0\) since
    \begin{equation*}
        \lim_{x \to 0} g(x)
        = \lim_{x \to 0} \frac{1}{\sin(x / 2)} - \frac{2}{x}
        = \lim_{x \to 0} \frac{x - 2\sin(x / 2)}{x \sin(x / 2)}
        = 0,
    \end{equation*}
    where the last equality follows by applying L'Hopital's rule twice.
    Therefore, since \(g(x)\) is continuous on a compact interval, it is Riemann integrable.
    So by the Riemann-Lebesgue Lemma,
    \begin{equation*}
        \lim_{N \to \infty} \int_{-\pi}^{\pi} g(x) \sin((N + 1 / 2)x) \ud x = 0.
    \end{equation*}
    Thus,
    \begin{equation*}
        \lim_{N \to \infty} \int_{-\pi}^{\pi} \sin((N + 1/2)x) \cdot \frac{2}{x} \ud x = 2\pi.
    \end{equation*}
    By dividing both sides by \(2\) and noting that the integrand is even, we see
    \begin{equation*}
        \lim_{N \to \infty} \int_{0}^{\pi} \frac{\sin((N + 1/2)x)}{x} \ud x = \frac{\pi}{2}.
    \end{equation*}
    Finally, a change of variables \(u = (N + 1/2)x\) followed by taking the limit yields
    \begin{equation*}
        \int_{0}^{\infty} \frac{\sin u}{u} \ud u = \frac{\pi}{2},
    \end{equation*}
    as desired.


\end{soln}
\end{exrc}
