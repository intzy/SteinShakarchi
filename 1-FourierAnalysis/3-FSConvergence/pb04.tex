%%% Problem 1.3.4
\begin{pb}[4]
    In this problem, we find the formula for the sum of the series
    \begin{equation*}
        \sum_{n = 1}^{\infty} \frac{1}{n^k}
    \end{equation*}
    where \(k\) is any even integer.
    These sums are expressed in terms of the Bernoulli numbers.

    Define the \emph{Bernoulli numbers} \(B_n\) by the formula
    \begin{equation*}
        \frac{z}{e^z - 1} = \sum_{n = 0}^{\infty} \frac{B_n}{n!} z^n.
    \end{equation*}
    \begin{enumerate}[(a)]
    \item
        Show that \(B_0 = 1\), \(B_1 = - 1 / 2\), \(B_2 = -1 / 6\),
        \(B_3 = 0\), \(B_4 = -1 / 30\), and \(B_5 = 0\).
    
    \item
        Show that for \(n \geq 1\) we have
        \begin{equation*}
            B_n = - \frac{1}{n + 1} \sum_{k = 0}^{n - 1}  \binom{n + 1}{k} B_k.
        \end{equation*}
    
    \item
        By writing
        \begin{equation*}
            \frac{z}{e^z - 1} = 1 - \frac{z}{2} + \sum_{n = 2}^{\infty} \frac{B_n}{n!}z^n,
        \end{equation*}
        shoe that \(B_n = 0\) if \(n\) is odd and \(> 1\).
        Also prove that
        \begin{equation*}
            z \cot z = 1 + \sum_{n = 1}^{\infty} (-1)^n \frac{2^{2n} B_{2n}}{(2n)!} z^{2n}.
        \end{equation*}
    
    \item
        The \emph{zeta function} is defined by
        \begin{equation*}
            \zeta(s) = \sum_{n = 1}^{\infty}  \frac{1}{n^s}, \quad \quad \text{for all \(s > 1\)}.
        \end{equation*}
        Deduce from the result in (c), and the expression for the cotangent function
        obtained in the problem, that
        \begin{equation*}
            x \cot x = 1 - 2 \sum_{m = 1}^{\infty} \frac{\zeta(2m)}{\pi^{2m}x^{2m}}.
        \end{equation*}
    
    \item
        Conclude that
        \begin{equation*}
            2 \zeta(2m) = (-1)^{m + 1} \frac{(2\pi)^{2m}}{(2m)!}B_{2m}.
        \end{equation*}
    \end{enumerate}

\begin{soln}
    \begin{enumerate}[(a)]
    \item
        Recall that \(e^z - 1 = \sum_{n = 1}^{\infty} z^n / n!\).
        Therefore,
        \begin{equation*}
            1 = \sum_{n = 0}^{\infty} \frac{1}{(n+1)!} z^n \sum_{n = 0}^{\infty} \frac{B_n}{n!} z^n.
        \end{equation*}
        Collecting like terms gives
        \begin{align*}
            1 &= \frac{1}{1!} \frac{B_0}{0!}, \\
            0 &= \frac{1}{2!} \frac{B_0}{0!} + \frac{1}{1!} \frac{B_1}{1!}, \\
            0 &= \frac{1}{3!} \frac{B_0}{0!} + \frac{1}{2!} \frac{B_1}{1!} + \frac{1}{1!} \frac{B_2}{2!}, \\
            0 &= \frac{1}{4!} \frac{B_0}{0!} + \frac{1}{3!} \frac{B_1}{1!}
                    + \frac{1}{2!} \frac{B_2}{2!} + \frac{1}{1!} \frac{B_3}{3!}, \\ 
            0 &= \frac{1}{5!} \frac{B_0}{0!} + \frac{1}{4!} \frac{B_1}{1!}
                    + \frac{1}{3!} \frac{B_2}{2!} + \frac{1}{2!} \frac{B_3}{3!}
                    + \frac{1}{1!} \frac{B_4}{4!}, \\
            0 &= \frac{1}{6!} \frac{B_0}{0!} + \frac{1}{5!} \frac{B_1}{1!}
                    + \frac{1}{4!} \frac{B_2}{2!} + \frac{1}{3!} \frac{B_3}{3!}
                    + \frac{1}{2!} \frac{B_4}{4!} + \frac{1}{1!} \frac{B_5}{5!}.
        \end{align*}
        Solving successively for \(B_0\) to \(B_5\) gives the desired result.

    \item
        From part (a), we showed that
        \begin{equation*}
            1 = \sum_{k = 0}^{\infty} \frac{z^k}{(k + 1)!} 
            \cdot \sum_{k = 0}^{\infty} \frac{B_k z^k}{k!}.
        \end{equation*}
        The infinite polynomial on the right-hand side of the above equation
        has zeros for all non-constant coefficients.
        Therefore, as we did in part (a),
        by collecting terms of the same degree in the above expression, we see
        \begin{equation*}
            0 = \sum_{k = 0}^{n} \frac{z^{n - k}}{(n - k + 1)!} \cdot \frac{B_k z^k}{k!}
        \end{equation*}
        for \(n \geq 0\).
        Solving for \(B_n\) above and factoring out \(z^n\) gives
        \begin{equation*}
            B_n = -n! \sum_{k = 0}^{n - 1} \frac{B_k}{(n - k + 1)! k!}.
        \end{equation*}
        Multiplying the terms in the sum by \((n + 1)\) shows
        \begin{equation*}
            B_n = -\frac{1}{n + 1} \sum_{k = 0}^{n - 1} \frac{(n + 1)! B_k}{(n - k + 1)! k !}
            = -\frac{1}{n + 1} \sum_{k = 0}^{n - 1} \binom{n + 1}{k} B_k.
        \end{equation*}
    
    \item
        Define
        \begin{equation*}
            f(z) = 1 + \sum_{k = 2}^{\infty} \frac{B_n}{n!} z_n.
        \end{equation*}
        Note that
        \begin{equation*}
            f(z) = \frac{z}{e^z - 1} + \frac{z}{2} = \frac{2z + ze^z - z}{2e^z - 2}
            = \frac{z}{2} \frac{e^z + 1}{e^z - 1}.
            = \frac{z}{2} \frac{e^{z / 2} + e^{- z / 2}}{e^{z / 2} - e^{-z / 2}}.
        \end{equation*}
        By the above equation, we see that \(f(z) = f(-z)\).
        Therefore, for each \(n \geq 2\), we must have \(B_n (-z)^n / n! = B_n z^n / n!\).
        So, if \(B_n\) is nonzero, then \((-1)^n z^n = z^n\),
        and this clearly holds only when \(n\) is even.
        Therefore, for odd \(n \geq 2\), we must have \(B_n = 0\).

        Next, by considering \(f(2iz)\), on one hand, we obtain
        \begin{equation*}
            f(2iz) = zi \frac{e^{iz} + e^{-iz}}{e^{iz} - e^{-iz}} = z \cot z.
        \end{equation*}
        On the other hand, by using the fact that \(B_n = 0\) for odd \(n \geq 2\), we see
        \begin{equation*}
            f(2iz) 
            = 1 + \sum_{k = 1}^{\infty} \frac{B_{2n}}{(2n)!} (2iz)^{2n}
            = 1 + \sum_{k = 1}^{\infty} (-1)^n \frac{2^{2n} B_{2n}}{(2n)!} z^{2n}.
        \end{equation*}
        Therefore,
        \begin{equation*}
            z \cot z = 1 + \sum_{k = 1}^{\infty} (-1)^n \frac{2^{2n} B_{2n}}{(2n)!} z^{2n}.
        \end{equation*}
    
    \item
        From the previous problem, we deduced that
        \begin{equation*}
            x \cot x = 1 + 2 \sum_{n = 1}^{\infty} \frac{x^2}{x^2 - n^2 \pi^2}.
        \end{equation*}
        Recall that \(1 / (1 - r) = \sum_{k}^{} r^k\) for \(|r| < 1\).
        Therefore, for \(|x| < \pi\),
        \begin{align*}
            x \cot x
            &= 1 + \frac{2x^2}{\pi^2} \sum_{n = 1}^{\infty} \frac{1}{n^2} \frac{1}{\frac{x^2}{n^2 \pi^2} - 1} \\
            &= 1 - \frac{2x^2}{\pi^2} \sum_{n = 1}^{\infty} \frac{1}{n^2} \sum_{m = 0}^{\infty} \lrb{\frac{x}{n\pi}}^{2m} \\
            &= 1 - 2 \sum_{n = 1}^{\infty} \sum_{m = 1}^{\infty} \frac{x^{2m}}{n^{2m}\pi^{2m}}.
        \end{align*}
        For any \(n \geq 1\), and \(|x| < \pi\), the sum
        \(\sum_{m = 1}^{\infty} \frac{x^{2m}}{n^{2m} \pi^{2m}}\) converges absolutely.
        So by the Weierstrass \(M\)-test,
        we can interchange the order of the infinite sums.
        Therefore
        \begin{equation*}
            x \cot x = 1 - 2 \sum_{m = 1}^{\infty} \sum_{n = 1}^{\infty} \frac{x^{2m}}{n^{2m}\pi^{2m}}
            = \sum_{m = 1}^{\infty} \frac{\zeta(2m)}{\pi^{2m}}x^{2m}.
        \end{equation*}
    
    \item
        From parts (c) and (d), we can conclude that for integers \(m \geq 1\), we have
        \begin{equation*}
            -\frac{2 \zeta(2m)}{\pi^{2m}} = (-1)^m \frac{2^{2m} B_{2m}}{(2m)!}.
        \end{equation*}
        Rearranging for \(2 \zeta(2m)\) gives the desired result.
        This gives a formula for \(\sum_{n = 1}^{\infty} 1 / n^k\) for any positive even \(k\).
        
    \end{enumerate}

\end{soln}
\end{pb}
