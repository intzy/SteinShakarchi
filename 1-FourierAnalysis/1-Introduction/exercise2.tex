%%% Exercise 2
\begin{exrc}[2]
    If \(z = x + iy\) is a complex number with \(x, y \in \R\),
    we define the \emph{complex conjugate} of \(z\) by
    \begin{equation*}
        \bar{z} = x + iy.
    \end{equation*}
    \begin{enumerate}[(a)]
    \item
        What is the geometric interpretation of \(\bar{z}\)?
    \item
        Show that \(|z|^2 = z \bar{z}\).
    \item
        Prove that if \(z\) belongs to the unit circle, then \(1 / z = \bar{z}\).
    \end{enumerate}

\begin{soln}
    \begin{enumerate}[(a)]
    \item
        Identifying \(z\) as a vector in \(\R^2\),
        the complex conjugate \(\bar{z}\)
        is \(z\) reflected across the real axis.
    \item
        A quick calculation shows
        \begin{equation*}
            z \bar{z} = (z + iy)(z - iy) = x^2 + y^2 = |z|^2.
        \end{equation*}
    \item
        Since \(z\) belongs on the unit circle,
        by part (b) we see
        \begin{equation*}
            z \bar{z} = 1.
        \end{equation*}
        Since \(z \neq 0\) by hypothesis,
        dividing by \(z\) gives the desired result.
    \end{enumerate}

\end{soln}
\end{exrc}
