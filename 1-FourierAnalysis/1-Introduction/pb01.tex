%%% Problem 1.1.1
\begin{pb}[1]
    Find a solution for the steady-state heat equation \(\triangle u = 0\) in the recatangle
    \(R = \set{(x, y): 0 \leq x \leq \pi, 0 \leq y \leq 1}\) that vanishes
    on the vertical sides of \(R\),
    and so that
    \begin{equation*}
        u(x, 0) = f_0(x) \quad \quad \text{and} \quad \quad u(x, 1) = f_1(x),
    \end{equation*}
    where \(f_0\) and \(f_1\) are initial data which fix the temperature distribution on the horizontal
    sides of the rectangle.

    Use separation of variables to show that if \(f_0\) and \(f_1\) have Fourier expansions
    \begin{equation*}
        f_0(x) = \sum_{k = 1}^{\infty} A_k \sin kx \quad \quad \text{and}
        f_1(x) = \sum_{k = 1}^{\infty} B_k \sin kx,
    \end{equation*}
    then
    \begin{equation*}
        u(x, y)
        = \sum_{k = 1}^{\infty} \lrb{\frac{\sinh(k(1 - y))}{\sinh k} A_k + \frac{\sinh ky}{\sinh k}B_k} \sin kx.
    \end{equation*}

\begin{soln}
    Using the technique of separation of variables, assume that \(u(x, y) = X(x) Y(y)\).
    Then \(\triangle u = 0\) is equivalent to saying that
    \begin{equation*}
        X'' Y + X Y'' = 0
        \quad \quad \text{or} \quad \quad
        -\frac{X''}{X} = \frac{Y''}{Y} = C,
    \end{equation*}
    where \(C\) is a constant.
    Since \(X\) is periodic, the constant \(C\) is nonnegative,
    so we can assume \(C = \lambda^2\), where \(\lambda\) is real.

    The differential equation
    \begin{equation*}
        X'' + \lambda^2 X = 0.
    \end{equation*}
    has, by exercise 6, the solution
    \begin{equation*}
        X(x) = a \cos \lambda x + b \sin \lambda x.
    \end{equation*}
    The initial condition \(X(0) = 0\) and \(X(\pi) = 0\)
    implies \(a = 0\) and \(\lambda \in \Z\) respectively,
    hence
    \begin{equation*}
        X(x) = b \sin \lambda x \quad \quad \lambda \in \Z.
    \end{equation*}
    Also, the differential equation
    \begin{equation*}
        Y'' - \lambda^2 Y = 0
    \end{equation*}
    has solution
    \begin{equation*}
        Y(y) = c_1 e^{\lambda y} + c_2 e^{-\lambda y},
    \end{equation*}
    where \(c_1\) and \(c_2\) are constants.


    By our assumption that \(u = XY\) and our results on \(X\) and \(Y\),
    we obtain for fixed \(k \in \Z\) that
    \begin{equation*}
        u(x, y) = \lrb{C_1 e^{k y} + C_2 e^{-k y}} \sin k x,
    \end{equation*}
    where \(C_i = c_i b\).
    Noting that the Laplacian is linear, we can superpose the above solutions into one general solution
    \begin{equation*}
        u(x, y) = \sum_{k = 1}^{\infty} \lrb{a_k e^{k y} + b_k e^{-k y}} \sin kx,
    \end{equation*}
    where \(a_k\) and \(b_k\) are constant in \(x\) and \(y\).
    We use the fact that \(\sin x\) is an odd function to simplify the two-sided infinite
    sum into a on-sided infinite sum

    Now we find the values of \(C_1 \) and \(C_2\).
    Using the initial conditions \(u(x, 0) = f_0\) and \(u(x, 1) = f_1\) yields
    \begin{equation*}
        a_k + b_k = A_k, \quad \quad a_k e^{k} + b_k e^{-k} = B_k.
    \end{equation*}
    Solving for \(a_k\) and \(b_k\) gives
    \begin{equation*}
        a_k = \frac{B_k - A_k e^{-k}}{2 \sinh k}, \quad \quad
        b_k = \frac{A_k e^{k} - B_k}{2 \sinh k}.
    \end{equation*}
    Therefore,
    \begin{align*}
        u(x, y)
        &= \sum_{k = 1}^{\infty} \lrb{\frac{B_k - A_k e^{-k}}{2 \sinh k} e^{ky} + \frac{A_k e^{k} - B_k}{2 \sinh k} e^{-ky}} \sin kx \\
        &= \sum_{k = 1}^{\infty} \lrb{\frac{\sinh(k(1 - y))}{\sinh k} A_k + \frac{\sinh ky}{\sinh k}B_k} \sin kx,
    \end{align*}
    as desired.


\end{soln}
\end{pb}
