%%% Exercise 1.1.11
\begin{exrc}[11]
    Show that if \(n \in \Z\),
    the only solutions of the differential equation
    \begin{equation*}
        r^2 F''(r) + r F'(r) - n^2 F(r) = 0,
    \end{equation*}
    which are twice differentiable when \(r > 0\),
    are given by the linear combinations \(r^n\) and \(r^{-n}\) when \(n \neq 0\),
    and 1 and \(\log r\) when \(n = 0\).

\begin{soln}
    We take care of the case \(n = 0\) first.
    In this case, we have the differential equation
    \begin{equation*}
        rF'' + F' = 0.
    \end{equation*}
    Integrating both sides yields
    \begin{equation*}
        F'(r) = c_1 / r,
    \end{equation*}
    where \(c_1\) is a constant.
    Integrating again yields
    \begin{equation*}
        F(r) = c_1 \log r + c_2,
    \end{equation*}
    as desired.

    Now assume \(n \neq 0\).
    Let \(F(r) = g(r) r^n\).  Then
    \begin{equation*}
        F' = g' r^n + n g' r^{n - 1}, \quad \quad
        F'' = g'' r^n + 2n g' r^{n - 1} + n(n - 1)g r^{n - 2}.
    \end{equation*}
    Plugging these into the original differential equation and simplifying yields
    \begin{equation*}
        r g'' + (2n + 1) g' r = 0.
    \end{equation*}
    Integrating both sides gives
    \begin{equation*}
        rg' + 2n g = c,
    \end{equation*}
    where \(c\) is a constant.
    Since \(n \neq 0\), the solution to this differential equation is
    \begin{equation*}
        g(r) = c_1 r^{-2n} + c_2,
    \end{equation*}
    where \(c_1\) and \(c_2\) are constant in \(r\).
    Hence
    \begin{equation*}
        F(r) = c_1 r^{-n} + c_2 r^n.
    \end{equation*}

\end{soln}
\end{exrc}
